%%
%% This is file `team-eagle-lazer-fang.tex',
%%
%% A modified copy of:
%%
%% sample-authordraft.text
%% 
%% IMPORTANT NOTICE:
%% 
%% For the copyright see the source file.
%% 
%% Any modified versions of this file must be renamed
%% with new filenames distinct from sample-manuscript.tex.
%% 
%% For distribution of the original source see the terms
%% for copying and modification in the file samples.dtx.
%% 
%% This generated file may be distributed as long as the
%% original source files, as listed above, are part of the
%% same distribution. (The sources need not necessarily be
%% in the same archive or directory.)
%%
%% The first command in your LaTeX source must be the \documentclass command.
%%%% Small single column format, used for CIE, CSUR, DTRAP, JACM, JDIQ, JEA, JERIC, JETC, PACMCGIT, TAAS, TACCESS, TACO, TALG, TALLIP (formerly TALIP), TCPS, TDSCI, TEAC, TECS, TELO, THRI, TIIS, TIOT, TISSEC, TIST, TKDD, TMIS, TOCE, TOCHI, TOCL, TOCS, TOCT, TODAES, TODS, TOIS, TOIT, TOMACS, TOMM (formerly TOMCCAP), TOMPECS, TOMS, TOPC, TOPLAS, TOPS, TOS, TOSEM, TOSN, TQC, TRETS, TSAS, TSC, TSLP, TWEB.
% \documentclass[acmsmall]{acmart}

%%%% Large single column format, used for IMWUT, JOCCH, PACMPL, POMACS, TAP, PACMHCI
% \documentclass[acmlarge,screen]{acmart}

%%%% Large double column format, used for TOG
% \documentclass[acmtog, authorversion]{acmart}

%%%% Generic manuscript mode, required for submission
%%%% and peer review
\documentclass[manuscript,screen,review]{acmart}
%% Fonts used in the template cannot be substituted; margin 
%% adjustments are not allowed.
%%
%% \BibTeX command to typeset BibTeX logo in the docs
\AtBeginDocument{%
  \providecommand\BibTeX{{%
    \normalfont B\kern-0.5em{\scshape i\kern-0.25em b}\kern-0.8em\TeX}}}

%% Rights management information.  This information is sent to you
%% when you complete the rights form.  These commands have SAMPLE
%% values in them; it is your responsibility as an author to replace
%% the commands and values with those provided to you when you
%% complete the rights form.
% \setcopyright{acmcopyright}
% \copyrightyear{2018}
% \acmYear{2018}
% \acmDOI{XXXXXXX.XXXXXXX}

% %% These commands are for a PROCEEDINGS abstract or paper.
% \acmConference[Conference acronym 'XX]{Make sure to enter the correct
%   conference title from your rights confirmation emai}{June 03--05,
%   2018}{Woodstock, NY}
% %
% %  Uncomment \acmBooktitle if th title of the proceedings is different
% %  from ``Proceedings of ...''!
% %
% \acmBooktitle{Woodstock '18: ACM Symposium on Neural Gaze Detection,
%  June 03--05, 2018, Woodstock, NY} 
% \acmPrice{15.00}
% \acmISBN{978-1-4503-XXXX-X/18/06}


%%
%% Submission ID.
%% Use this when submitting an article to a sponsored event. You'll
%% receive a unique submission ID from the organizers
%% of the event, and this ID should be used as the parameter to this command.
%%\acmSubmissionID{123-A56-BU3}

%%
%% The majority of ACM publications use numbered citations and
%% references.  The command \citestyle{authoryear} switches to the
%% "author year" style.
%%
%% If you are preparing content for an event
%% sponsored by ACM SIGGRAPH, you must use the "author year" style of
%% citations and references.
%% Uncommenting
%% the next command will enable that style.
%%\citestyle{acmauthoryear}

%%
%% end of the preamble, start of the body of the document source.
\begin{document}

%%
%% The "title" command has an optional parameter,
%% allowing the author to define a "short title" to be used in page headers.
\title{Trustworthiness of AI decision making tools in high-risk applications.}

%%
%% The "author" command and its associated commands are used to define
%% the authors and their affiliations.
%% Of note is the shared affiliation of the first two authors, and the
%% "authornote" and "authornotemark" commands
%% used to denote shared contribution to the research.
\author{Alex Davies}
\affiliation{%
  \institution{Interactive AI CDT, University of Bristol}
  \country{UK}
}
\author{Daniel Collins}
\affiliation{%
  \institution{Interactive AI CDT, University of Bristol}
  \country{UK}
}
\author{Isabella Degen}
\affiliation{%
  \institution{Interactive AI CDT, University of Bristol}
  \country{UK}
}
\author{Jonathan Erskine}
\affiliation{%
  \institution{Interactive AI CDT, University of Bristol}
  \country{UK}
}
\author{Matt Clifford}
\affiliation{%
  \institution{Interactive AI CDT, University of Bristol}
  \country{UK}
}
\author{Ronja Spaniel}
\affiliation{%
  \institution{Interactive AI CDT, University of Bristol}
  \country{UK}
}

%%
%% By default, the full list of authors will be used in the page
%% headers. Often, this list is too long, and will overlap
%% other information printed in the page headers. This command allows
%% the author to define a more concise list
%% of authors' names for this purpose.
\renewcommand{\shortauthors}{Team Eagle Lazer Fang, et al.}

%%
%% The abstract is a short summary of the work to be presented in the
%% article.
\begin{abstract}
Abstract
 Include motivation and Contribution
\end{abstract}

%%
%% The code below is generated by the tool at http://dl.acm.org/ccs.cfm.
%% Please copy and paste the code instead of the example below.
%%
% \begin{CCSXML}
% <ccs2012>
%  <concept>
%   <concept_id>10010520.10010553.10010562</concept_id>
%   <concept_desc>Computer systems organization~Embedded systems</concept_desc>
%   <concept_significance>500</concept_significance>
%  </concept>
%  <concept>
%   <concept_id>10010520.10010575.10010755</concept_id>
%   <concept_desc>Computer systems organization~Redundancy</concept_desc>
%   <concept_significance>300</concept_significance>
%  </concept>
%  <concept>
%   <concept_id>10010520.10010553.10010554</concept_id>
%   <concept_desc>Computer systems organization~Robotics</concept_desc>
%   <concept_significance>100</concept_significance>
%  </concept>
%  <concept>
%   <concept_id>10003033.10003083.10003095</concept_id>
%   <concept_desc>Networks~Network reliability</concept_desc>
%   <concept_significance>100</concept_significance>
%  </concept>
% </ccs2012>
% \end{CCSXML}

% \ccsdesc[500]{Computer systems organization~Embedded systems}
% \ccsdesc[300]{Computer systems organization~Redundancy}
% \ccsdesc{Computer systems organization~Robotics}
% \ccsdesc[100]{Networks~Network reliability}

%%
%% Keywords. The author(s) should pick words that accurately describe
%% the work being presented. Separate the keywords with commas.
\keywords{XAI, trustworthiness, AI, computer automation}

%%
%% This command processes the author and affiliation and title
%% information and builds the first part of the formatted document.
\maketitle

\newpage
\section{Introduction}
Include motivation


\begin{itemize}
    \item AI used in many use cases (give examples of high risk etc) 
    \item For successful integration of AI, users and public need to trust it. 
    \item Law came into that decisions for banks etc needed to be explained to the user, the right to explanation, e.g GDPR
    \item this brought about interest in XAI (Why XAI has been seen as important (Issues with this - black box models, not always human able to intervene/oversee))
    \item Lots of work has been done into XAI techniques to give `trust' in AI.
    \item XAI examples to show off the amount of work been done  - survey papers and specific examples? LIME? FATF?
    \item Issues with XAI - critics (to give some doubt over XAI actually giving trust and telling us how AI works)
    \item Is the current work in XAI just useful for machine learning practictoners? 
    \item What is best for the users/ what do users actually want? Has the explaination law confused us of what users actually want/need?
    \item We want to explore what trust actually is in relation to AI. Is it XAI? Is it performance? Is it counterfactual results? (are they more intuitive/what people would like to know)?
    \item There has been the topic of trust between humans and machines before (1987 paper). Which indicates that there needs to be trust not just performance.
    \item There is no current user study of is XAI gives more trust or what actually gives trust for user. And how this changes depending on demographic (AI expert, high stakes decision maker, general public) This is what motivates our study.

    % \item medical domain - paper Alex found on simple AI in a medical context out-performing doctors
\end{itemize}


% AI is becoming widespread in multiple domains to automate and aid decision making tasks. (Give some examples).  However, many AI algorithms are inherently black box. This means that the inner workings and reasoning that an AI uses to arrive at a decision is unknown. It is argued that in some application areas[cite], such as ad servers, that black box models are sufficient.



\section{Related Work}
Literature Review

Rough overview
\begin{itemize}
    \item critics of XAI - is it what we really need for trust?, does it really given an explanation for a non AI expert? 
    \item trustworthiness? does XAI make us trust? Lipton paper (Mythos) is trustworthiness resolved XAI or is it performance?
\end{itemize}

\subsection{Available Techniques for Explainability and Interpretability}
There has been a previous wave of explainability in control systems as well as AI in the late 80ies: Explaining Control Strategies in Problem Solving\cite{Chandrasekaran1989}

There's been a plethora of interpretability and explainability techniques developed in recent years - surge since 2016. There are many survey papers that describe the different techniques and try to come up with a framework to organise them:
\begin{itemize}
    \item Explainable AI: A Review of Machine Learning Interpretability Methods\cite{Linardatos2021}
    \item Notions of explainability and evaluation approaches for explainable artificial intelligence\cite{Vilone2021}
    \item A Comprehensive Taxonomy for Explainable Artificial Intelligence: A Systematic Survey of Surveys on Methods and Concepts\cite{Schwalbe2021}
    \item Explainable Artificial Intelligence: a Systematic Review\cite{Vilone2020}
    \item A Survey of Methods for Explaining Black Box Models\cite{Guidotti2018a}
    \item Peeking Inside the Black-Box: A Survey on Explainable Artificial Intelligence (XAI)\cite{Adadi2018}
    \item A Multidisciplinary Survey and Framework for Design and Evaluation of Explainable AI Systems\cite{Mohseni2021}
    \item Explainable Deep Learning: A Field Guide for the Uninitiated\cite{Ras2020}
    \item How Case-Based Reasoning Explains Neural Networks: A Theoretical Analysis of XAI Using Post-Hoc Explanation-by-Example from a Survey of ANN-CBR Twin-Systems\cite{Keane2019}
    \item Improving fairness in machine learning systems: What do industry practitioners need?\cite{Holstein2019}
    \item Good Counterfactuals and Where to Find Them: A Case-Based Technique for Generating Counterfactuals for Explainable AI (XAI)\cite{Keane2020} 
    \item Contrastive explanation: A structural-model approach\cite{Miller2021}
\end{itemize}

\subsection{Paper that argue the need of XAI}
\begin{itemize}
    \item XAI-Explainable artificial intelligence\cite{Gunning2019} Quote "Explainability is essential for users to effectively understand, trust, and manage powerful artificial intelligence applications."
\end{itemize}


\subsection{Shortfalls and Critics of current XAI}
The Mythos of model interpretability\cite{Lipton2018}: critic on if explainability and interpretability can be achieved and does it lead to trust or not? Critics of the vague definition of the terms used.

Do interpretable models achieve their intended effects: Manipulating and Measuring Model Interpretability\cite{Goldstein2021}

Explanation in Artificial Intelligence: Insights from the Social Sciences\cite{Miller2019}
%https://www.connectedpapers.com/main/e89dfa306723e8ef031765e9c44e5f6f94fd8fda/Explanation-in-Artificial-Intelligence%3A-Insights-from-the-Social-Sciences/graph

Towards A Rigorous Science of Interpretable Machine Learning\cite{Doshi-Velez2017}


\subsection{Trust in Machines}
Trust is an important factor in facilitating Human Computer Interaction. While trust remains hard to define, there's a significant body of work around 'trust in automated systems' for high-stakes decision making.

Trust Between Humans and Machines, and the Design of Decision Aids\cite{Muir1987}
"The concept of trust is a critical one in the design of decision support systems. A
decision aid, no matter how sophisticated or "intelligent" it may be, may be rejected
by a decision maker who does not trust it, and so its potential benefits to system
performance will be lost."
%https://www.connectedpapers.com/main/c58081eec27d282757efbb35b023d9c81c699a5e/Trust-Between-Humans-and-Machines%2C-and-the-Design-of-Decision-Aids/graph

Foundations for an Empirically Determined Scale of Trust in Automated Systems\cite{Jian2000}
An empirically developed measure of trust. The framework developed in this paper has continued to be used in more recent research to measure trust.
%https://www.connectedpapers.com/main/d3f274a407ab237cfffc822dae103b108b9bdfff/Foundations-for-an-Empirically-Determined-Scale-of-Trust-in-Automated-Systems/graph

Trust in Automation: Integrating Empirical Evidence on Factors That Influence Trust \cite{Hoff2015} Results: "three layers of variability in human–automation trust (dispositional trust, situational trust, and learned trust), which we organize into a model. We propose design recommendations for creating trustworthy automation and identify environmental conditions that can affect the strength of the relationship between trust and reliance."

A Meta-Analysis of Factors Affecting Trust in Human-Robot Interaction\cite{Hancock2011} Results: "Factors related to the robot itself, specifically, its performance, had the greatest current association with trust, and environmental factors were moderately associated. There was little evidence for effects of human-related factors."

In AI We Trust Incrementally: a Multi-layer Model of Trust to Analyze Human-Artificial Intelligence Interactions\cite{Ferrario2020}

Trust and Trust-Engineering in Artificial Intelligence Research: Theory and Praxis\cite{Chen2021}

The relationship between trust in AI and trustworthy machine learning technologies\cite{Toreini2020}


\section{Methodology}
Describe how we did user research
How are we going to evaluate our findings

\section{Design}


\section{Findings}
Describe our findings from user research
Evaluation of our findings

\section{Discussion}
Contribution we've made


\section{Conclusion}



%%
%% The acknowledgments section is defined using the "acks" environment
%% (and NOT an unnumbered section). This ensures the proper
%% identification of the section in the article metadata, and the
%% consistent spelling of the heading.
\begin{acks}
Acknowledgments
\end{acks}

%%
%% The next two lines define the bibliography style to be used, and
%% the bibliography file.
\bibliographystyle{ACM-Reference-Format}
\bibliography{elf}

%%
%% If your work has an appendix, this is the place to put it.
\appendix

\section{Optional Supplementary Material}

\subsection{Part One}

\subsection{Part Two}


\end{document}
\endinput
%%
%% End of file.
